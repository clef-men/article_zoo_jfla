\section{\Zoo in practice}
\label{sec:zoo}

\section{\Zoo in practice}
\label{sec:zoo}

\section{\Zoo in practice}
\label{sec:zoo}

\section{\Zoo in practice}
\label{sec:zoo}

\input{figures/zoo}

In this section, we give an overview of the framework.
We also provide a \href{https://github.com/clef-men/zoo-demo}{minimal example}\footnote{\url{https://github.com/clef-men/zoo-demo}} demonstrating its use.

\paragraph{Language.}

The core of \Zoo is \ZooLang: an untyped, ML-like, imperative, concurrent programming language that is fully formalized in \Coq.
Its semantics has been designed to match \OCaml's.

\ZooLang comes with a program logic based on \Iris : reasoning rules expressed in separation logic (including rules for the different constructs of the language) along with \Coq tactics that integrate into the \Iris proof mode~\cite{DBLP:conf/popl/KrebbersTB17,DBLP:journals/pacmpl/KrebbersJ0TKTCD18}.
In addition, it supports \Diaframe~\cite{DBLP:conf/pldi/MulderKG22}, enabling proof automation.

The \ZooLang syntax is given in \cref{fig:zoo}\footnote{More precisely, it is the syntax of the surface language, including many \Coq notations.}, omitting mutually recursive toplevel functions that are treated specifically.
Expressions include standard constructs like booleans, integers, anonymous functions (that may be recursive), \mintinline{ocaml}{let} bindings, sequence, unary and binary operators, conditionals, \mintinline{ocaml}{for} loops, tuples.
In any expression, one can refer to a \Coq term representing a \ZooLang value (of type \mintinline{coq}{val}) using its \Coq identifier.
\ZooLang is a deeply embedded language: variables (bound by functions and \mintinline{ocaml}{let}) are quoted, represented as strings.

Data constructors (immutable memory blocks) are supported through two constructs : $\texttt{§}C$ represents a constant constructor (\eg $\texttt{§}\texttt{None}$), $\texttt{‘} C\ \texttt{(} e_1 \texttt{,} \dots \texttt{,} e_n \texttt{)}$ represents a non-constant constructor (\eg $\texttt{‘} \texttt{Some( } e \texttt{ )}$).
Unlike \OCaml, \ZooLang has projections of the form $e \texttt{.<} \mathit{proj} \texttt{>}$ (\eg $\texttt{(} e_1 \texttt{,} e_2 \texttt{).<1>}$), that can be used to obtain a specific component of a tuple or data constructor.
\ZooLang supports shallow pattern matching (patterns cannot be nested) on data constructors with an optional fallback case.

Mutable memory blocks are constructed using either the untagged record syntax $\texttt{\{} e_1 \texttt{,} \dots \texttt{,} e_n \texttt{\}}$ or the tagged record syntax $\texttt{‘} C\ \texttt{\{} e_1 \texttt{,} \dots \texttt{,} e_n \texttt{\}}$.
Reading a record field can be performed using $e \texttt{.\{} \mathit{fld} \texttt{\}}$ and writing to a record field using $e_1\ \texttt{<-\{} \mathit{fld} \texttt{\}}\ e_2$.
Pattern matching can also be used on mutable tagged blocks provided that cases do not bind anything---in other words, only the tag is examined, no memory access is performed.
References are also supported through the usual constructs : $\texttt{ref}\ e$ creates a reference, $\texttt{!} e$ reads a reference and $e_1\ \texttt{<-}\ e_2$ writes into a reference.
The syntax seemingly does not include constructs for arrays but they are supported through the \mintinline{ocaml}{Array} standard module (\eg \texttt{array\_make}).

Parallelism is mainly supported through the \mintinline{ocaml}{Domain} standard module (\eg \texttt{domain\_spawn}).
Special constructs (\texttt{Xchg}, \texttt{CAS}, \texttt{FAA}), described in \cref{sec:atomic}, are used to model atomic references.

The \texttt{Proph} and \texttt{Resolve} constructs are used to model \emph{prophecy variables}~\cite{DBLP:journals/pacmpl/JungLPRTDJ20}, as described in \cref{sec:prophecy}.

Finally, \texttt{Reveal} is a special source construct that we introduce to handle physical equality.
We demystify it in \cref{sec:physical_equality}.

\paragraph{Translation from \OCaml to \ZooLang.}

While \ZooLang lives in \Coq, we want to verify \OCaml programs.
To connect them, we provide a tool to automatically translate \OCaml source files\footnote{Actually, \texttt{ocaml2zoo} processes binary annotation files (\texttt{.cmt} files).} into \Coq files containing \ZooLang code: \texttt{ocaml2zoo}.
This tool can process entire \texttt{dune} projects, including many libraries.

The supported \OCaml fragment includes: shallow \mintinline{ocaml}{match}, ADTs, records, inline records, atomic record fields, unboxed types, toplevel mutually recursive functions.

As an example of what \texttt{ocaml2zoo} can generate, the \mintinline{ocaml}{push} function from \cref{sec:introduction} is translated into:

\begin{minted}{coq}
Definition stack_push : val :=
  rec: "push" "t" "v" =>
    let: "old" := !"t" in
    let: "new_" := "v" :: "old" in
    if: ~ CAS "t".[contents] "old" "new_" then (
      domain_yield () ;; "push" "t" "v"
    ).
\end{minted}

\paragraph{Specifications and proofs.}

Once the translation to \ZooLang is done, the user can write specifications and prove them in \Iris.
For instance, the specification of the \mintinline{ocaml}{stack_push} function could be:

\begin{minted}{coq}
Lemma stack_push_spec t ι v :
  <<< stack_inv t ι | ∀∀ vs, stack_model t vs >>>
    stack_push t v @ ↑ι
  <<< stack_model t (v :: vs) | RET (); True >>>.
Proof. ... Qed.
\end{minted}

Here, we use a \emph{logically atomic specification}~\cite{DBLP:conf/ecoop/PintoDG14}, which has been proven~\cite{DBLP:journals/pacmpl/BirkedalDGJST21} to be equivalent to \emph{linearizability}~\cite{DBLP:journals/toplas/HerlihyW90} in sequentially consistent memory models.

Similarly to \href{https://en.wikipedia.org/wiki/Hoare_logic}{Hoare triples}, the two assertions inside curly brackets represent the precondition and postcondition for the caller.
For this particular operation, the postcondition is trivial.
The $\mathrm{stack \mathhyphen inv}\ t$ precondition is the stack invariant.
Intuitively, it asserts that $t$ is a valid concurrent stack.
More precisely, it enforces a set of logical constraints---a concurrent protocol---that $t$ must respect at all times.

The other two assertions inside angle brackets represent the \emph{atomic precondition} and \emph{atomic postcondition}.
They specify the linearization point of the operation: during the execution of \texttt{stack\_push}, the abstract state of the stack held by $\mathrm{stack \mathhyphen model}$ is atomically updated from $\mathit{vs}$ to $v :: \mathit{vs}$; in other words, $v$ is atomically pushed at the top of the stack.


In this section, we give an overview of the framework.
We also provide a \href{https://github.com/clef-men/zoo-demo}{minimal example}\footnote{\url{https://github.com/clef-men/zoo-demo}} demonstrating its use.

\paragraph{Language.}

The core of \Zoo is \ZooLang: an untyped, ML-like, imperative, concurrent programming language that is fully formalized in \Coq.
Its semantics has been designed to match \OCaml's.

\ZooLang comes with a program logic based on \Iris : reasoning rules expressed in separation logic (including rules for the different constructs of the language) along with \Coq tactics that integrate into the \Iris proof mode~\cite{DBLP:conf/popl/KrebbersTB17,DBLP:journals/pacmpl/KrebbersJ0TKTCD18}.
In addition, it supports \Diaframe~\cite{DBLP:conf/pldi/MulderKG22}, enabling proof automation.

The \ZooLang syntax is given in \cref{fig:zoo}\footnote{More precisely, it is the syntax of the surface language, including many \Coq notations.}, omitting mutually recursive toplevel functions that are treated specifically.
Expressions include standard constructs like booleans, integers, anonymous functions (that may be recursive), \mintinline{ocaml}{let} bindings, sequence, unary and binary operators, conditionals, \mintinline{ocaml}{for} loops, tuples.
In any expression, one can refer to a \Coq term representing a \ZooLang value (of type \mintinline{coq}{val}) using its \Coq identifier.
\ZooLang is a deeply embedded language: variables (bound by functions and \mintinline{ocaml}{let}) are quoted, represented as strings.

Data constructors (immutable memory blocks) are supported through two constructs : $\texttt{§}C$ represents a constant constructor (\eg $\texttt{§}\texttt{None}$), $\texttt{‘} C\ \texttt{(} e_1 \texttt{,} \dots \texttt{,} e_n \texttt{)}$ represents a non-constant constructor (\eg $\texttt{‘} \texttt{Some( } e \texttt{ )}$).
Unlike \OCaml, \ZooLang has projections of the form $e \texttt{.<} \mathit{proj} \texttt{>}$ (\eg $\texttt{(} e_1 \texttt{,} e_2 \texttt{).<1>}$), that can be used to obtain a specific component of a tuple or data constructor.
\ZooLang supports shallow pattern matching (patterns cannot be nested) on data constructors with an optional fallback case.

Mutable memory blocks are constructed using either the untagged record syntax $\texttt{\{} e_1 \texttt{,} \dots \texttt{,} e_n \texttt{\}}$ or the tagged record syntax $\texttt{‘} C\ \texttt{\{} e_1 \texttt{,} \dots \texttt{,} e_n \texttt{\}}$.
Reading a record field can be performed using $e \texttt{.\{} \mathit{fld} \texttt{\}}$ and writing to a record field using $e_1\ \texttt{<-\{} \mathit{fld} \texttt{\}}\ e_2$.
Pattern matching can also be used on mutable tagged blocks provided that cases do not bind anything---in other words, only the tag is examined, no memory access is performed.
References are also supported through the usual constructs : $\texttt{ref}\ e$ creates a reference, $\texttt{!} e$ reads a reference and $e_1\ \texttt{<-}\ e_2$ writes into a reference.
The syntax seemingly does not include constructs for arrays but they are supported through the \mintinline{ocaml}{Array} standard module (\eg \texttt{array\_make}).

Parallelism is mainly supported through the \mintinline{ocaml}{Domain} standard module (\eg \texttt{domain\_spawn}).
Special constructs (\texttt{Xchg}, \texttt{CAS}, \texttt{FAA}), described in \cref{sec:atomic}, are used to model atomic references.

The \texttt{Proph} and \texttt{Resolve} constructs are used to model \emph{prophecy variables}~\cite{DBLP:journals/pacmpl/JungLPRTDJ20}, as described in \cref{sec:prophecy}.

Finally, \texttt{Reveal} is a special source construct that we introduce to handle physical equality.
We demystify it in \cref{sec:physical_equality}.

\paragraph{Translation from \OCaml to \ZooLang.}

While \ZooLang lives in \Coq, we want to verify \OCaml programs.
To connect them, we provide a tool to automatically translate \OCaml source files\footnote{Actually, \texttt{ocaml2zoo} processes binary annotation files (\texttt{.cmt} files).} into \Coq files containing \ZooLang code: \texttt{ocaml2zoo}.
This tool can process entire \texttt{dune} projects, including many libraries.

The supported \OCaml fragment includes: shallow \mintinline{ocaml}{match}, ADTs, records, inline records, atomic record fields, unboxed types, toplevel mutually recursive functions.

As an example of what \texttt{ocaml2zoo} can generate, the \mintinline{ocaml}{push} function from \cref{sec:introduction} is translated into:

\begin{minted}{coq}
Definition stack_push : val :=
  rec: "push" "t" "v" =>
    let: "old" := !"t" in
    let: "new_" := "v" :: "old" in
    if: ~ CAS "t".[contents] "old" "new_" then (
      domain_yield () ;; "push" "t" "v"
    ).
\end{minted}

\paragraph{Specifications and proofs.}

Once the translation to \ZooLang is done, the user can write specifications and prove them in \Iris.
For instance, the specification of the \mintinline{ocaml}{stack_push} function could be:

\begin{minted}{coq}
Lemma stack_push_spec t ι v :
  <<< stack_inv t ι | ∀∀ vs, stack_model t vs >>>
    stack_push t v @ ↑ι
  <<< stack_model t (v :: vs) | RET (); True >>>.
Proof. ... Qed.
\end{minted}

Here, we use a \emph{logically atomic specification}~\cite{DBLP:conf/ecoop/PintoDG14}, which has been proven~\cite{DBLP:journals/pacmpl/BirkedalDGJST21} to be equivalent to \emph{linearizability}~\cite{DBLP:journals/toplas/HerlihyW90} in sequentially consistent memory models.

Similarly to \href{https://en.wikipedia.org/wiki/Hoare_logic}{Hoare triples}, the two assertions inside curly brackets represent the precondition and postcondition for the caller.
For this particular operation, the postcondition is trivial.
The $\mathrm{stack \mathhyphen inv}\ t$ precondition is the stack invariant.
Intuitively, it asserts that $t$ is a valid concurrent stack.
More precisely, it enforces a set of logical constraints---a concurrent protocol---that $t$ must respect at all times.

The other two assertions inside angle brackets represent the \emph{atomic precondition} and \emph{atomic postcondition}.
They specify the linearization point of the operation: during the execution of \texttt{stack\_push}, the abstract state of the stack held by $\mathrm{stack \mathhyphen model}$ is atomically updated from $\mathit{vs}$ to $v :: \mathit{vs}$; in other words, $v$ is atomically pushed at the top of the stack.


In this section, we give an overview of the framework.
We also provide a \href{https://github.com/clef-men/zoo-demo}{minimal example}\footnote{\url{https://github.com/clef-men/zoo-demo}} demonstrating its use.

\paragraph{Language.}

The core of \Zoo is \ZooLang: an untyped, ML-like, imperative, concurrent programming language that is fully formalized in \Coq.
Its semantics has been designed to match \OCaml's.

\ZooLang comes with a program logic based on \Iris : reasoning rules expressed in separation logic (including rules for the different constructs of the language) along with \Coq tactics that integrate into the \Iris proof mode~\cite{DBLP:conf/popl/KrebbersTB17,DBLP:journals/pacmpl/KrebbersJ0TKTCD18}.
In addition, it supports \Diaframe~\cite{DBLP:conf/pldi/MulderKG22}, enabling proof automation.

The \ZooLang syntax is given in \cref{fig:zoo}\footnote{More precisely, it is the syntax of the surface language, including many \Coq notations.}, omitting mutually recursive toplevel functions that are treated specifically.
Expressions include standard constructs like booleans, integers, anonymous functions (that may be recursive), \mintinline{ocaml}{let} bindings, sequence, unary and binary operators, conditionals, \mintinline{ocaml}{for} loops, tuples.
In any expression, one can refer to a \Coq term representing a \ZooLang value (of type \mintinline{coq}{val}) using its \Coq identifier.
\ZooLang is a deeply embedded language: variables (bound by functions and \mintinline{ocaml}{let}) are quoted, represented as strings.

Data constructors (immutable memory blocks) are supported through two constructs : $\texttt{§}C$ represents a constant constructor (\eg $\texttt{§}\texttt{None}$), $\texttt{‘} C\ \texttt{(} e_1 \texttt{,} \dots \texttt{,} e_n \texttt{)}$ represents a non-constant constructor (\eg $\texttt{‘} \texttt{Some( } e \texttt{ )}$).
Unlike \OCaml, \ZooLang has projections of the form $e \texttt{.<} \mathit{proj} \texttt{>}$ (\eg $\texttt{(} e_1 \texttt{,} e_2 \texttt{).<1>}$), that can be used to obtain a specific component of a tuple or data constructor.
\ZooLang supports shallow pattern matching (patterns cannot be nested) on data constructors with an optional fallback case.

Mutable memory blocks are constructed using either the untagged record syntax $\texttt{\{} e_1 \texttt{,} \dots \texttt{,} e_n \texttt{\}}$ or the tagged record syntax $\texttt{‘} C\ \texttt{\{} e_1 \texttt{,} \dots \texttt{,} e_n \texttt{\}}$.
Reading a record field can be performed using $e \texttt{.\{} \mathit{fld} \texttt{\}}$ and writing to a record field using $e_1\ \texttt{<-\{} \mathit{fld} \texttt{\}}\ e_2$.
Pattern matching can also be used on mutable tagged blocks provided that cases do not bind anything---in other words, only the tag is examined, no memory access is performed.
References are also supported through the usual constructs : $\texttt{ref}\ e$ creates a reference, $\texttt{!} e$ reads a reference and $e_1\ \texttt{<-}\ e_2$ writes into a reference.
The syntax seemingly does not include constructs for arrays but they are supported through the \mintinline{ocaml}{Array} standard module (\eg \texttt{array\_make}).

Parallelism is mainly supported through the \mintinline{ocaml}{Domain} standard module (\eg \texttt{domain\_spawn}).
Special constructs (\texttt{Xchg}, \texttt{CAS}, \texttt{FAA}), described in \cref{sec:atomic}, are used to model atomic references.

The \texttt{Proph} and \texttt{Resolve} constructs are used to model \emph{prophecy variables}~\cite{DBLP:journals/pacmpl/JungLPRTDJ20}, as described in \cref{sec:prophecy}.

Finally, \texttt{Reveal} is a special source construct that we introduce to handle physical equality.
We demystify it in \cref{sec:physical_equality}.

\paragraph{Translation from \OCaml to \ZooLang.}

While \ZooLang lives in \Coq, we want to verify \OCaml programs.
To connect them, we provide a tool to automatically translate \OCaml source files\footnote{Actually, \texttt{ocaml2zoo} processes binary annotation files (\texttt{.cmt} files).} into \Coq files containing \ZooLang code: \texttt{ocaml2zoo}.
This tool can process entire \texttt{dune} projects, including many libraries.

The supported \OCaml fragment includes: shallow \mintinline{ocaml}{match}, ADTs, records, inline records, atomic record fields, unboxed types, toplevel mutually recursive functions.

As an example of what \texttt{ocaml2zoo} can generate, the \mintinline{ocaml}{push} function from \cref{sec:introduction} is translated into:

\begin{minted}{coq}
Definition stack_push : val :=
  rec: "push" "t" "v" =>
    let: "old" := !"t" in
    let: "new_" := "v" :: "old" in
    if: ~ CAS "t".[contents] "old" "new_" then (
      domain_yield () ;; "push" "t" "v"
    ).
\end{minted}

\paragraph{Specifications and proofs.}

Once the translation to \ZooLang is done, the user can write specifications and prove them in \Iris.
For instance, the specification of the \mintinline{ocaml}{stack_push} function could be:

\begin{minted}{coq}
Lemma stack_push_spec t ι v :
  <<< stack_inv t ι | ∀∀ vs, stack_model t vs >>>
    stack_push t v @ ↑ι
  <<< stack_model t (v :: vs) | RET (); True >>>.
Proof. ... Qed.
\end{minted}

Here, we use a \emph{logically atomic specification}~\cite{DBLP:conf/ecoop/PintoDG14}, which has been proven~\cite{DBLP:journals/pacmpl/BirkedalDGJST21} to be equivalent to \emph{linearizability}~\cite{DBLP:journals/toplas/HerlihyW90} in sequentially consistent memory models.

Similarly to \href{https://en.wikipedia.org/wiki/Hoare_logic}{Hoare triples}, the two assertions inside curly brackets represent the precondition and postcondition for the caller.
For this particular operation, the postcondition is trivial.
The $\mathrm{stack \mathhyphen inv}\ t$ precondition is the stack invariant.
Intuitively, it asserts that $t$ is a valid concurrent stack.
More precisely, it enforces a set of logical constraints---a concurrent protocol---that $t$ must respect at all times.

The other two assertions inside angle brackets represent the \emph{atomic precondition} and \emph{atomic postcondition}.
They specify the linearization point of the operation: during the execution of \texttt{stack\_push}, the abstract state of the stack held by $\mathrm{stack \mathhyphen model}$ is atomically updated from $\mathit{vs}$ to $v :: \mathit{vs}$; in other words, $v$ is atomically pushed at the top of the stack.


In this section, we give an overview of the framework.
We also provide a \href{https://github.com/clef-men/zoo-demo}{minimal example}\footnote{\url{https://github.com/clef-men/zoo-demo}} demonstrating its use.

\paragraph{Language.}

The core of \Zoo is \ZooLang: an untyped, ML-like, imperative, concurrent programming language that is fully formalized in \Coq.
Its semantics has been designed to match \OCaml's.

\ZooLang comes with a program logic based on \Iris : reasoning rules expressed in separation logic (including rules for the different constructs of the language) along with \Coq tactics that integrate into the \Iris proof mode~\cite{DBLP:conf/popl/KrebbersTB17,DBLP:journals/pacmpl/KrebbersJ0TKTCD18}.
In addition, it supports \Diaframe~\cite{DBLP:conf/pldi/MulderKG22}, enabling proof automation.

The \ZooLang syntax is given in \cref{fig:zoo}\footnote{More precisely, it is the syntax of the surface language, including many \Coq notations.}, omitting mutually recursive toplevel functions that are treated specifically.
Expressions include standard constructs like booleans, integers, anonymous functions (that may be recursive), \mintinline{ocaml}{let} bindings, sequence, unary and binary operators, conditionals, \mintinline{ocaml}{for} loops, tuples.
In any expression, one can refer to a \Coq term representing a \ZooLang value (of type \mintinline{coq}{val}) using its \Coq identifier.
\ZooLang is a deeply embedded language: variables (bound by functions and \mintinline{ocaml}{let}) are quoted, represented as strings.

Data constructors (immutable memory blocks) are supported through two constructs : $\texttt{§}C$ represents a constant constructor (\eg $\texttt{§}\texttt{None}$), $\texttt{‘} C\ \texttt{(} e_1 \texttt{,} \dots \texttt{,} e_n \texttt{)}$ represents a non-constant constructor (\eg $\texttt{‘} \texttt{Some( } e \texttt{ )}$).
Unlike \OCaml, \ZooLang has projections of the form $e \texttt{.<} \mathit{proj} \texttt{>}$ (\eg $\texttt{(} e_1 \texttt{,} e_2 \texttt{).<1>}$), that can be used to obtain a specific component of a tuple or data constructor.
\ZooLang supports shallow pattern matching (patterns cannot be nested) on data constructors with an optional fallback case.

Mutable memory blocks are constructed using either the untagged record syntax $\texttt{\{} e_1 \texttt{,} \dots \texttt{,} e_n \texttt{\}}$ or the tagged record syntax $\texttt{‘} C\ \texttt{\{} e_1 \texttt{,} \dots \texttt{,} e_n \texttt{\}}$.
Reading a record field can be performed using $e \texttt{.\{} \mathit{fld} \texttt{\}}$ and writing to a record field using $e_1\ \texttt{<-\{} \mathit{fld} \texttt{\}}\ e_2$.
Pattern matching can also be used on mutable tagged blocks provided that cases do not bind anything---in other words, only the tag is examined, no memory access is performed.
References are also supported through the usual constructs : $\texttt{ref}\ e$ creates a reference, $\texttt{!} e$ reads a reference and $e_1\ \texttt{<-}\ e_2$ writes into a reference.
The syntax seemingly does not include constructs for arrays but they are supported through the \mintinline{ocaml}{Array} standard module (\eg \texttt{array\_make}).

Parallelism is mainly supported through the \mintinline{ocaml}{Domain} standard module (\eg \texttt{domain\_spawn}).
Special constructs (\texttt{Xchg}, \texttt{CAS}, \texttt{FAA}), described in \cref{sec:atomic}, are used to model atomic references.

The \texttt{Proph} and \texttt{Resolve} constructs are used to model \emph{prophecy variables}~\cite{DBLP:journals/pacmpl/JungLPRTDJ20}, as described in \cref{sec:prophecy}.

Finally, \texttt{Reveal} is a special source construct that we introduce to handle physical equality.
We demystify it in \cref{sec:physical_equality}.

\paragraph{Translation from \OCaml to \ZooLang.}

While \ZooLang lives in \Coq, we want to verify \OCaml programs.
To connect them, we provide a tool to automatically translate \OCaml source files\footnote{Actually, \texttt{ocaml2zoo} processes binary annotation files (\texttt{.cmt} files).} into \Coq files containing \ZooLang code: \texttt{ocaml2zoo}.
This tool can process entire \texttt{dune} projects, including many libraries.

The supported \OCaml fragment includes: shallow \mintinline{ocaml}{match}, ADTs, records, inline records, atomic record fields, unboxed types, toplevel mutually recursive functions.

As an example of what \texttt{ocaml2zoo} can generate, the \mintinline{ocaml}{push} function from \cref{sec:introduction} is translated into:

\begin{minted}{coq}
Definition stack_push : val :=
  rec: "push" "t" "v" =>
    let: "old" := !"t" in
    let: "new_" := "v" :: "old" in
    if: ~ CAS "t".[contents] "old" "new_" then (
      domain_yield () ;; "push" "t" "v"
    ).
\end{minted}

\paragraph{Specifications and proofs.}

Once the translation to \ZooLang is done, the user can write specifications and prove them in \Iris.
For instance, the specification of the \mintinline{ocaml}{stack_push} function could be:

\begin{minted}{coq}
Lemma stack_push_spec t ι v :
  <<< stack_inv t ι | ∀∀ vs, stack_model t vs >>>
    stack_push t v @ ↑ι
  <<< stack_model t (v :: vs) | RET (); True >>>.
Proof. ... Qed.
\end{minted}

Here, we use a \emph{logically atomic specification}~\cite{DBLP:conf/ecoop/PintoDG14}, which has been proven~\cite{DBLP:journals/pacmpl/BirkedalDGJST21} to be equivalent to \emph{linearizability}~\cite{DBLP:journals/toplas/HerlihyW90} in sequentially consistent memory models.

Similarly to \href{https://en.wikipedia.org/wiki/Hoare_logic}{Hoare triples}, the two assertions inside curly brackets represent the precondition and postcondition for the caller.
For this particular operation, the postcondition is trivial.
The $\mathrm{stack \mathhyphen inv}\ t$ precondition is the stack invariant.
Intuitively, it asserts that $t$ is a valid concurrent stack.
More precisely, it enforces a set of logical constraints---a concurrent protocol---that $t$ must respect at all times.

The other two assertions inside angle brackets represent the \emph{atomic precondition} and \emph{atomic postcondition}.
They specify the linearization point of the operation: during the execution of \texttt{stack\_push}, the abstract state of the stack held by $\mathrm{stack \mathhyphen model}$ is atomically updated from $\mathit{vs}$ to $v :: \mathit{vs}$; in other words, $v$ is atomically pushed at the top of the stack.
