\section{\Zoo in practice}
\label{sec:zoo}

\begin{figure}[tp]
\centering
\begin{tabular}{llcl}
    \Coq term &
    $t$
  \\
    constructor &
    $C$
  \\
    projection &
    $\mathit{proj}$
  \\
    record field &
    $\mathit{fld}$
  \\
    identifier &
    $s, f$
    & $\in$ &
    $\mathrm{String}$
  \\
    integer &
    $n$
    & $\in$ &
    $\mathbb{Z}$
  \\
    boolean &
    $b$
    & $\in$ &
    $\mathbb{B}$
  \\
    binder &
    $x$
    & $\Coloneqq$ &
    $\texttt{<>} \mid s$
  \\
    unary operator &
    $\oplus$
    & $\Coloneqq$ &
    $\texttt{\raisebox{0.5ex}{\texttildelow}} \mid \texttt{-}$
  \\
    binary operator &
    $\otimes$
    & $\Coloneqq$ &
    $\texttt{+} \mid \texttt{-} \mid \texttt{*} \mid \texttt{`quot`} \mid \texttt{`rem`}$
  \\
    && | &
    $\texttt{<=} \mid \texttt{<} \mid \texttt{>=} \mid \texttt{>} \mid \texttt{=} \mid \texttt{≠} \mid \texttt{==} \mid \texttt{!=}$
  \\
    && | &
    $\texttt{and} \mid \texttt{or}$
  \\
    expression &
    $e$
    & $\Coloneqq$ &
    $t \mid s \mid \texttt{\#} n \mid \texttt{\#} b$
  \\
    && | &
    $\texttt{fun:}\ x_1 \dots x_n\ \texttt{=>}\ e \mid \texttt{rec:}\ f\ x_1 \dots x_n\ \texttt{=>}\ e$
  \\
   && | &
   $\texttt{let:}\ x\ \texttt{:=}\ e_1\ \texttt{in}\ e_2 \mid e_1\ \texttt{;;}\ e_2$
  \\
    && | &
    $\texttt{let:}\ f\ x_1 \dots x_n\ \texttt{:=}\ e_1\ \texttt{in}\ e_2 \mid \texttt{letrec:}\ f\ x_1 \dots x_n\ \texttt{:=}\ e_1\ \texttt{in}\ e_2$
  \\
    && | &
    $\texttt{let:}\ \texttt{‘} C\ x_1 \dots x_n\ \texttt{:=}\ e_1\ \texttt{in}\ e_2 \mid \texttt{let:}\ x_1 \texttt{,} \dots \texttt{,} x_n\ \texttt{:=}\ e_1\ \texttt{in}\ e_2$
  \\
    && | &
    $\oplus e \mid e_1 \otimes e_2$
  \\
    && | &
    $\texttt{if:}\ e_0\ \texttt{then}\ e_1\ (\texttt{else}\ e_2)^? \mid \texttt{ifnot:}\ e_0\ \texttt{then}\ e_1$
  \\
    && | &
    $\texttt{for:}\ x\ \texttt{:=}\ e_1\ \texttt{to}\ e_2\ \texttt{begin}\ e_3\ \texttt{end}$
  \\
    && | &
    $\texttt{§}C \mid \texttt{‘} C\ \texttt{(} e_1 \texttt{,} \dots \texttt{,} e_n \texttt{)} \mid \texttt{(} e_1 \texttt{,} \dots \texttt{,} e_n \texttt{)} \mid e \texttt{.<} \mathit{proj} \texttt{>}$
  \\
    && | &
    $\texttt{[]} \mid e_1\ \texttt{::}\ e_2$
  \\
    && | &
    $\texttt{Alloc}\ e_1\ e_2 \mid \texttt{ref}\ e \mid \texttt{!} e \mid e_1\ \texttt{<-}\ e_2$
  \\
    && | &
    $\texttt{‘} C\ \texttt{\{} e_1 \texttt{,} \dots \texttt{,} e_n \texttt{\}} \mid \texttt{\{} e_1 \texttt{,} \dots \texttt{,} e_n \texttt{\}} \mid e \texttt{.\{} \mathit{fld} \texttt{\}} \mid e_1\ \texttt{<-\{} \mathit{fld} \texttt{\}}\ e_2$
  \\
    && | &
    $\texttt{Reveal}\ e \mid \texttt{GetTag}\ e \mid \texttt{GetSize}\ e$
  \\
    && | &
    $\texttt{match:}\ e_0\ \texttt{with}\ \mathit{br}_1 \texttt{|} \dots \texttt{|}\ \mathit{br}_n\ (\texttt{|\_}\ (\texttt{as}\ s)^?\ \texttt{=>}\ e)^?\ \texttt{end}$
  \\
    && | &
    $\texttt{Fork}\ e \mid \texttt{Yield}$
  \\
    && | &
    $e \texttt{.[} \mathit{fld} \texttt{]} \mid \texttt{Xchg}\ e_1\ e_2 \mid \texttt{CAS}\ e_1\ e_2\ e_3 \mid \texttt{FAA}\ e_1\ e_2$
  \\
    && | &
    $\texttt{Proph} \mid \texttt{Resolve}\ e_0\ e_1\ e_2$
  \\
    branch &
    $\mathit{br}$
    & $\Coloneqq$ &
    $C\ (x_1 \dots x_n)^?\ (\texttt{as}\ s)^?\ \texttt{=>}\ e$
  \\
    && | &
    $\texttt{[]}\ (\texttt{as}\ s)^?\ \texttt{=>}\ e \mid x_1\ \texttt{::}\ x_2\ (\texttt{as}\ s)^?\ \texttt{=>}\ e$
  \\
    toplevel value &
    $v$
    & $\Coloneqq$ &
    $t \mid \texttt{\#} n \mid \texttt{\#} b$
  \\
    && | &
    $\texttt{fun:}\ x_1 \dots x_n\ \texttt{=>}\ e \mid \texttt{rec:}\ f\ x_1 \dots x_n\ \texttt{=>}\ e$
  \\
    && | &
    $\texttt{§}C \mid \texttt{‘} C\ \texttt{(} v_1 \texttt{,} \dots \texttt{,} v_n \texttt{)} \mid \texttt{(} v_1 \texttt{,} \dots \texttt{,} v_n \texttt{)}$
  \\
    && | &
    $\texttt{[]} \mid v_1\ \texttt{::}\ v_2$
\end{tabular}
\caption{\Zoo syntax (omitting mutually recursive toplevel functions)}
\label{fig:zoo}
\end{figure}

Before describing the salient features of our language, \Zoo, in \cref{sec:features}, we give an overview of the framework.

\paragraph{From \OCaml to \Zoo.}

First, \OCaml source files are translated into \Zoo by the \texttt{ocaml2zoo} tool.
The \Zoo syntax is given in \cref{fig:zoo}\footnote{More precisely, it is the syntax of the surface language, including many \Coq notations.}, omitting mutually recursive toplevel functions that are treated specifically.
Essentially, \Zoo is an untyped, ML-like, imperative, concurrent programming language.
The supported \OCaml fragment includes: shallow \mintinline{ocaml}{match}, ADTs, records, inline records, atomic record fields, unboxed types, toplevel mutually recursive functions.

For instance, the \mintinline{ocaml}{push} function from \cref{sec:introduction} is translated into:

\begin{minted}{coq}
Definition stack_push : val :=
  rec: "push" "t" "v" =>
    let: "old" := !"t" in
    let: "new_" := "v" :: "old" in
    ifnot: CAS "t".[contents] "old" "new_" then (
      Yield ;;
      "push" "t" "v"
    ).
\end{minted}

\paragraph{Specifications and proofs.}

Second, the user writes specifications for the translated functions and prove them using the \Iris proof mode~\cite{DBLP:journals/pacmpl/KrebbersJ0TKTCD18}.

For instance, the specification for the \mintinline{ocaml}{stack_push} function would be:

\begin{minted}{coq}
Lemma stack_push_spec t ι v :
  <<<
    stack_inv t ι
  | ∀∀ vs, stack_model t vs
  >>>
    stack_push t v @ ↑ι
  <<<
    stack_model t (v :: vs)
  | RET (); True
  >>>.
Proof.
  ...
Qed.
\end{minted}

Here, we use a \emph{logically atomic specification}~\cite{DBLP:conf/ecoop/PintoDG14}, which has been proven~\cite{DBLP:journals/pacmpl/BirkedalDGJST21} to be equivalent to \emph{linearizability}~\cite{DBLP:journals/toplas/HerlihyW90} in sequentially consistent memory models.
